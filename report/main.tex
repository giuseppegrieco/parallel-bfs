\documentclass{article}

\usepackage[utf8]{inputenc}
\usepackage[a4paper]{geometry}
\usepackage[boxed]{algorithm2e}
\usepackage{amsmath}
\usepackage{hyperref}
\usepackage{adjustbox}
\usepackage{xcolor}
\selectcolormodel{gray}

\begin{document}
\vspace*{\fill}
\begin{center}
    {\Huge{Design and implementation of parallel breadth-first search} }\\[2mm]
    \vspace{6.5mm}
    
    {\large \textbf{Project Report}}\\[5pt]
    \vspace{15mm}

    {\large \textbf{Academic year 2020-2021}}\\
    {\large \textit{Submitted for the examination of Parallel and distributed systems: paradigms and models}}\\[5pt]
    \vspace{6.5mm}

    {\large By}\\
    \vspace{3mm}
    {\large\textbf{Giuseppe Grieco}}\\
    {\large g.grieco6@studenti.unipi.it}\\

    \vspace{11mm}
    \includegraphics[scale=0.07]{./images/logo.png}\\
    \vspace{11mm}

    {\large Department of Computer Science\\
    \textsc{University of Pisa, Italy}}\\
    \vspace{15mm}

    {\large{July, 2021}}
\end{center}
\vspace*{\fill}
\newpage

\tableofcontents
\newpage


\section{Introduction}
\subsection{Problem statement}
The bread-first search is an algorithm visiting the graph in amplitude. It starts from a node, 
often called the root or source node, and continues visiting all its descendants level
by level, whereas the $i$-th level will contain all the nodes at a distance $i$ from the root. 
The assumption underlying all the work is that the input to the problem is a direct and 
acyclic graph. In addition, for the sake of clarity, the notation used is summarized below:
\\

Let $\mathcal{G} = (V, E)$, $|V| = n$ is the number of node and $|E| = m$ is the number of edges. 
For all the node $v \in V$:
\begin{itemize}
    \item $\mathcal{N}(v) = \{u : (u, v) \in E\}$ is the neighborhood of $v$;
    \item $k_{in}(v) = |\{e: e=(u, v) \in E\}|$ is the in-degree of $v$;
    \item $k_{out}(v) = |\mathcal{N}(v)|$ is the out-degree of $v$;
    \item $k(v) = k_{in}(v) + k_{out}(v)$ is the degree of $v$.
    \item $d(u, v)$ is the distance from $u$ to $v$, i.e. the minimum path from $u$ to $v$.
\end{itemize}
Using the node-focus notation above,
 it is possible to define general properties for the graph:
\begin{itemize}
    \item $\bar{k}$ is the average degree;
    \item $\bar{d}$ is the average distance;
    \item $d_{max}$ is the diameter.
\end{itemize}

\subsection{Proposed solution}
Before entering in the algorithmic details. let's first introduce the data structure used.
There are many ways to represent a graph among these the main ones are adjacency list and adjacency
matrix. The choices among them is mainly a matter of the usage of the adjacency information and the 
expected nature of the graph. As for every node $v$ of the graph, induced by the root, it will be necessary
to go through each node $u \in \mathcal{N}(v)$, the adjacency list is way more efficient since for each
node $v$ the listing of $\mathcal{N}(v)$ takes a time proportional to $k_{out}(v)$, which is thus optimal.
Moreover, if the expected input of the algorithm are "real" graphs than since they are very sparse, the representation as a adjacency list is way 
more efficient in terms of space complexity.  
\subsubsection{Sequential version}
The sequential version
\subsubsection{Parallel version}
\section{Analysis and measurements}
\section{Conclusion}
\end{document}